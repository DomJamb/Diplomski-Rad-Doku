\documentclass[diplomskirad]{fer}
% Dodaj opciju upload za generiranje konačne verzije koja se učitava na FERWeb
% Add the option upload to generate the final version which is uploaded to FERWeb


%--- PODACI O RADU / THESIS INFORMATION ----------------------------------------

% Naslov na hrvatskom jeziku / Title in Croatian
\naslov{Varijacijsko učenje na zašumljenim oznakama}

% Broj rada / Thesis number
\brojrada{872}

% Autor / Author
\author{Dominik Jambrović}

% Mentor 
\mentor{prof.\ dr.\ sc.\ Siniša Šegvić}

% Datum rada na hrvatskom jeziku / Date in Croatian
\datum{lipanj, 2025.}

%-------------------------------------------------------------------------------


\begin{document}


% Naslovnica se automatski generira / Titlepage is automatically generated
\maketitle


%--- ZADATAK / THESIS ASSIGNMENT -----------------------------------------------

% Zadatak se ubacuje iz vanjske datoteke / Thesis assignment is included from external file
% Upiši ime PDF datoteke preuzete s FERWeb-a / Enter the filename of the PDF downloaded from FERWeb
\zadatak{hr_0036534818_94.pdf}


%--- ZAHVALE / ACKNOWLEDGMENT --------------------------------------------------

\begin{zahvale}
  % Ovdje upišite zahvale / Write in the acknowledgment
  Zahvale!
\end{zahvale}


% Odovud započinje numeriranje stranica / Page numbering starts from here
\mainmatter


% Sadržaj se automatski generira / Table of contents is automatically generated
\tableofcontents


%--- UVOD / INTRODUCTION -------------------------------------------------------
\chapter{Uvod}
\label{pog:uvod}
  

Duboki modeli koriste se u brojnim aspektima naše svakodnevice.\ Pri razvoju i učenju modela, pažnju prije svega posvećujemo performansama na neviđenim podatcima - želimo naučiti modele koji dobro generaliziraju.\ 
Drugim riječima, želimo da modeli daju ispravna predviđanja za viđene, ali i za neviđene podatke.\ Ovime osiguravamo da naša rješenja imaju primjenu i van laboratorijskih uvjeta u kojima se uče.\

U procesu razvoja modela za određeni zadatak strojnog učenja, osim odabira arhitekture, algoritma učenja i hiperparametara, veliku ulogu igraju podatci na kojima učimo.\ 
Općenito govoreći, prikupljanje i označavanje podataka jedan je od najskupljih dijelova procesa razvoja rješenja za nekih problem.\ 
Važno je da prikupljeni podatci što realističnije predstavljaju stvarne situacije s kojima će se naš model susretati tj.\ da distribucija podataka odgovara stvarnoj distribuciji situacija koje prikazuju.\ 
Dodatno, pokazuje se da duboki modeli uz dovoljan kapacitet mogu naučiti ispravno predviđati oznake čak i za nasumično označene podatke~\cite{zhang2016understanding}, tako da je veoma važno da su prikupljeni podatci što točnije označeni.\

Područje računalnog vida~\cite{voulodimos2018deep} bavi se razvojem algoritama i modela za brojne zadatke raspoznavanja i razumijevanja slika.\ Najčešći zadatak je klasifikacija slika - model na ulazu dobiva sliku, a na izlazu treba predvidjeti razred koji odgovara ulaznom primjeru.\ 
Iako postoje brojni skupovi slikovnih podataka koji se mogu koristiti za učenje i evaluaciju modela, za konkretne zadatke u većini slučajeva trebamo prikupiti i označiti vlastite slike.\ 
Pritom postoji nekoliko čestih opasnosti: prisutnost zatrovanih~\cite{biggio2012poisoning} ili zašumljenih~\cite{gupta2019dealing} podataka.\ 

\pagebreak

Kada govorimo o trovanju podataka, maliciozni agent u skup podataka dodaje zatrovane podatke s ciljem manipulacije izlaza naučenog modela za određene ulaze.\ 
S druge strane, anotator podataka bez zlih namjera određenim podatcima može pridijeliti netočne oznake, time dodajući zašumljene podatke u skup.\ 
Kroz vrijeme, razvili su se brojni algoritmi za obranu modela od zatrovanih~\cite{li2021anti, huang2022backdoor, zhang2023backdoor} odnosno zašumljenih~\cite{liu2022robust, chen2024imprecise} podataka.\ Ipak, većina radova se fokusira na samo jedan od ovih problema, a ne na razvoj algoritma koji se može nositi s oba problema.\ 

Cilj ovog rada je reproducirati i poboljšati rezultate okvira za obranu od zatrovanih podataka imena VIBE~\cite{sabolic2025seal}.\ 
Osim ovoga, cilj je i primijeniti VIBE na problem zašumljenih podataka.\ Pritom VIBE evaluiramo na nekoliko čestih vrsta trovanja odnosno zašumljivanja podataka kako bi se osigurala robusnost okvira.\ 
Dodatno, cilj je usporediti VIBE sa stanjem tehnike (engl.\ \textit{state of the art - SotA}) za problem zašumljenih podataka.\


%-------------------------------------------------------------------------------
\chapter{Problem zatrovanih podataka}
\label{pog:zatrovani}


%-------------------------------------------------------------------------------
\chapter{Problem zašumljenih podataka}
\label{pog:zasumljeni}


%-------------------------------------------------------------------------------
\chapter{Samonadzirano učenje}
\label{pog:samonadzirano}


%-------------------------------------------------------------------------------
\chapter{Algoritam maksimizacije očekivanja}
\label{pog:em_algoritam}


%-------------------------------------------------------------------------------
\chapter{Transportni problem}
\label{pog:transport}


%-------------------------------------------------------------------------------
\chapter{VIBE}
\label{pog:vibe}


%-------------------------------------------------------------------------------
\chapter{Skup podataka}
\label{pog:skup}


%-------------------------------------------------------------------------------
\chapter{Eksperimenti}
\label{pog:eksperimenti}


%--- ZAKLJUČAK / CONCLUSION ----------------------------------------------------
\chapter{Zaključak}
\label{pog:zakljucak}


%--- LITERATURA / REFERENCES ---------------------------------------------------

% Literatura se automatski generira iz zadane .bib datoteke / References are automatically generated from the supplied .bib file
% Upiši ime BibTeX datoteke bez .bib nastavka / Enter the name of the BibTeX file without .bib extension
\bibliography{literatura}


%--- SAŽETAK / ABSTRACT --------------------------------------------------------

% Sažetak na hrvatskom
\begin{sazetak}
  Sažetak...
\end{sazetak}

\begin{kljucnerijeci}
  prva ključna riječ; druga ključna riječ; treća ključna riječ
\end{kljucnerijeci}


\end{document}
